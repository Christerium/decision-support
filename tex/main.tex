\documentclass{article}
\usepackage{graphicx} % Required for inserting images
\usepackage{amsmath}

\title{Decision Support in Production, Logistics and Supply Chain}
\author{Arezoo Amiri \and Christof Brandstetter \and Tamara Ertl}
\date{\today}

\begin{document}

\maketitle

% \section{Introduction}
% We only need to add edges with G.add edge(node1, node2, capacity)
% It automatically adds nodes

% Pyvis does not add nodes automatically

% net.toggle physics(True), sometimes makes sense to set to False, It makes the graph draggable

% Generate $C_l$ cut where $\gamma = 1$ and generate $C_m$ where $\gamma = u$.
% Then calculate the bisection to obtain a new $\hat{\gamma}$.
% Then generate a cut with $\hat{\gamma}$ and do this until we find a cut that has less than $B$ large edges.

% % Transform the problem to max clique, 
% \section{Notation}
% \begin{itemize}
%     \item A cut is a partition $V  = S \cup T$ of the nodes of G such that $s \in S$ and $t \in T$
%     \item An arc $r \in E$ is in a cut $C = (S,T)$ if $\alpha(r) \in S$ and $\omega(r) \in T$
%     \item Arcs having capacity $u$ are called large arcs
%     \item Arcs having capacity 1 are called small arcs
%     \item $q(C) := \#$(large arcs in $C$)
%     \item $p(C) := \#$(small arcs in $C$)
% \end{itemize}
% \section{What's to do?}
% \begin{itemize}
%     \item Transform the digraph into a NFI graph
%           \begin{itemize}
%               \item Create a $s$-node
%               \item Create a node for every edge in the digraph
%               \item Create edges from $s$ to every \emph{edge-node} $i_x$ with capacity 2m
%               \item Create a node for every node in the digraph
%               \item Create edges from every \emph{edge-node} to every \emph{node-node} $j_1$ which is connected by the edges with capacity m
%               \item Create a node $t$ and connect the \emph{node-nodes} with the $t$ node by edges with capacity m
%               \item where $m = |E|$
%           \end{itemize}
%     \item Implement the bisection algorithm for NFI
%           \begin{itemize}
%               \item Generate a minimum cut $C_m$ (minimum Cut to the original graph) and get $q(C_m)$ and $p(C_m)$
%               \item Generate a least cut $C_l$ (minimum cut of G with the capacity of all arcs set to 1) and get $q(C_l)$ and $p(C_l)$
%               \item Calculate $\hat{\gamma}$ and generate the cut with capacity-$\hat{\gamma}$-min-cut $\hat{C}$
%               \item If $cap^\gamma(\hat{C}) \leq cap^\gamma(C_l)$ and $q(\hat{C}) \notin \{q(C_l),q(C_m)\}$ then generate two new cuts using $((C_l, \hat{C}), (\hat{C}, C_m))$
%               \item otherwise return $\{C_1, C_2\}$
%               \item The minimum cut is the smallest set of edges that, when removed, disconnects the graph into two disjoint subgraphs.
%               \item Identify the arcs $R \in C$ which are removed from the graph
%               \item $C_m$ is optimal if it contains at least $B$ large arcs with $val(C_m) = \text{cap of arcs in cut} - \text{cap of removed arcs}$
%               \item Assume: Any minimum cut in $G$ contains at most $B-1$ large arcs
%               \item $C_l$ denotes a least cut in G, i.e.,  a cut with least possible number of arcs. Then $C_l$ is optimal if it contains at most $B$ large arcs
%               \item Assume: Any least cut in $G$ contains at least $B+1$ large arcs
%           \end{itemize}
%     \item Find in the NFI graph a strategy for NFI with budget $B = |E| - \binom{K}{2} $ that has value $K*m$ to get a clique of size $K$
%     \item Transform it back to find the max-clique
%     \item Use the bisection algorithm for NFI to find large cliques for the benchmark set
%           \begin{itemize}
%               \item Suspect, that we have to do this for different K and raise the K's
%           \end{itemize}
% \end{itemize}

% $R \subseteq E$ is a solution to the u-NFI problem with objective value val(R) which equals the capacity of a minimum s-t-cut in the graph $G_R$.
% Minimum s-t-cut of a graph $G_R$ is a minimum cut (cut with least number of arcs, which disjoints s and t) and the capacity of this cut equals the maximum
% flow of the grpah. (Value of a cut is the sum of the capacities of the arcs in the cut) -> cut != removing arcs

% So network flow interdiction problem is about finding a subset of arcs $R$ that minimizes the maximum flow from s to t, where all arcs have different capacities
% (do not need to be different but there are several different ones).

% The u-NFI has small (1) and large (u) arcs. Here we need to keep in mind, that the val(C) = val($R_C$) = sum of capacity of the $B$ largest arcs


% If think the idea of algorithm to find a good solution to the u-NFI is that the we have to compute q-min-cuts
% (cut with smallest capacity in graph G amongst all cuts with exactly q large arcs) for different values of q. So find the minimum cut with minimal capacity
% and exactly q large arcs for different values of q and caluclating a q-min-cut is NP-hard.

% So we calculate minimum cuts when varying the capacity of the large arcs to obtain cuts with different numbers of q.

% If $C^\gamma$ is a capacity-$\gamma$-min-cut for some $\gamma \geq 1$, then it is a q-min-cut for $q = q(C^\gamma)$. This means that if we have a capacity-
% $\gamma$-min-cut $C^\gamma$ we have found a q-min-cut with q = q($C^\gamma$) or alternatively, by finding a minimum cut for a certain $\gamma$ we obtain a q-min-cut
% where q equals the number of large arcs in our minimum cut.

% Therefore, we start with $C_l$ and $C_m$ as the q obtained from these are bounds on the optimal q. Then we pick the next $\gamma$ to check for by doing this
% bisection. If this cut has a lower capacity than our lower bound (is below our two lines) we do the bisection again for $C_l$ $\hat{C}$ and $\hat{C}$ $C_m$.
% An this we do again and again until $\hat{C}$ has a higher capacity.

% To finish this up, we will not find a single cut, but rather a set of two cuts which will give us a lower and upper bound on the optimal value.


% \section{Additional thoughts}
% \begin{enumerate}
%     \item capacity-$\gamma$-min-cut $(C^\gamma)$ are $q$-min-cut for $q = q(C^\gamma)$
%     \item $q(C^{\gamma_1}) \geq q(C^{\gamma_2})$ for $1 \geq \gamma_1 \geq \gamma_2 \geq u$ $\rightarrow \#$ of large arcs is decreasing for increasing $\gamma$
%     \item We have bounds on q with $q(C_l)$ and $q(C_m)$
%     \item Given two cuts $C_1$ and $C_2$, the next $\hat{\gamma}$ is chosen by the value, when $cap(C_1) = cap(C_2)$ in the graph $G^{\hat{\gamma}}$
%     \item If $cap(C^{\hat{\gamma}}) < cap(C_1) = cap(C_2)$ in $G^{\hat{\gamma}}$ bisection is called again for $C_1$ and $\hat{C}$ and $C_2$ and $\hat{C}$
%     \item Otherwise the recursion ends and returns $C_1$ and $C_2$
%     \item We can use digrpahs for finding the cut as it does not matter for finding the cut
% \end{enumerate}
% The target here is to find a capacity-$\gamma$-min-cut with high $\gamma$ (equals low \# of large arcs) wihle reducing a low cut capacity.


% \section{New Formulation}
% $\epsilon_P$ is indicatiing if path $P = (s, i, j, t)$ is interdicted. A path is interdicted if arc $(s,i)$ is of the path is interdicted.
% $\beta$ still has the same meaning. Set $S^j$ is a set of all paths that are going through vertice node $j$. And set $S^i$ is a set of paths
% going through edge nodes $i$.
% \begin{alignat*}{5}
%       & \min &  & \sum_{(i,j) \in E} &  & u_{(i,j)}\beta_{(i,j)} &  &                                 &  &                                                 \\
%       & s.t. &  &                    &  & \beta_{(j,t)}          &  & \geq (1-\epsilon_P) \quad \quad &  & \forall P \in S^j \; \forall j \in N_{vertices} \\
%       &      &  & \sum_{(i,j)\in E}  &  & \gamma_{(i,j)}         &  & \leq B                          &  & \forall P \in S^i \; \forall i \in N_{edges}
% \end{alignat*}


% \section*{Old Formulation}

% \begin{align*}
%      \min &  & \sum_{(i,j) \in E} u_{(i,j)}\beta_{(i,j)}            &        &                                   \\
%      s.t. &  & \alpha_i - \alpha_j + \beta_{(i,j)} + \gamma_{(i,j)} & \geq 0 & \forall (i,j) \in E               \\
%           &  & \alpha_t - \alpha_s                                  & \geq 1 &                                   \\
%           &  & \sum_{(i,j) \in E} \gamma_{(i,j)}                    & \leq B &                                   \\
%           &  & \gamma_{(i,j)} = 0                                   &        & \forall (i,j) \notin first layer  \\
%           &  & \beta{(i,j)} = 0                                     &        & \forall (i,j) \notin thrid layer
% \end{align*}

\section{Introduction}
Our intial assignment was to solve the max-clique problem by transforming a graph into a network flow interdiction (NFI) graph and then using the bisection algorithm to find minmal cuts in the NFI graph. These minimal cuts, then would be used to find a clique of size $K$ in the original graph. This approach quickly turned out to be problematic, as $e^{'}/e^{*} \leq n^{*}/n^{'}$, where the prime indicates not in the clique and the star indicates in the clique and $n$ is the number of nodes in the graph and $e$ is the number of edges. This ratio is not true in many instances. Therefore, we decided to change our approach and use the integer programming formulation of the max-clique problem from Wood to solve the problem. As the max-clique problem is NP-hard and the formulation of Wood was not providing us with the hoped results, we decided to change the formulation to a new one, which we will present in this report.

\section{Wood Formulation}
$\alpha_i = 1$ for $i$ on the t side of the cut and $\alpha_i = 0$ for $i$ on the s side of the cut. $\beta_{(i,j)} = 1$ if arc $(i,j)$ is a forward arc across the cut but is not to be broke and $\gamma_{(i,j)} = 1$ if arc $(i,j)$ is a forward arc across the cut and is to be broken, all other $\beta_{(i,j)}$ and $\gamma_{(i,j)}$ are 0.  
\begin{align*}
     \min &  & \sum_{(i,j) \in A} u_{(i,j)}\beta_{(i,j)}                                      \\
     s.t. &  & \alpha_i - \alpha_j + \beta_{ij} + \gamma_{ij} & \leq 0 & \forall (i,j) \in A  \\
          &  & \alpha_{t} - \alpha_{s}                        & \geq 1 &                      \\
          &  & \sum_{(i,j) \in A} r_{ij}\gamma_{ij}           & \leq R &                      \\
          &  & \alpha_i \in \{0,1\}                           &        & \forall i \in N      \\
          &  & \beta_{ij}, \gamma_{ij} \in \{0,1\}            &        & \forall (i,j) \in A
\end{align*}
The objective function minimizes the cost of all forward arcs that are not to be broken, hence the maximum flow from $s$ to $t$ is minimized. Only in the case of a forward arc the first constraint is binding, as for $(\alpha_i = 1, \alpha_j = 0)$ both $\beta_{ij}$ and $\gamma_{ij}$ are 0. As well as for the case of $\alpha_i = \alpha_j$. Therefore, we only need to consider forward arcs ($\alpha_i = 0, \alpha_j = 1$). The first constraint then ensures that either $\beta_{ij}$ or $\gamma_{ij}$ is 1, so it is either a forward arc that is not to be broken or a forward arc that is to be broken. The second constraint ensures that $t$ and $s$ cannot be on the same side of the cut. The third constraint ensures that the total number of forward arcs that are to be broken is less than or equal to $R$.

\section{Our End formulation}
Let $N^E$ be the set of "edge nodes" and $N^V$ be the set of "vertice nodes" in a network flow interdiction graph (NFI), where an edge node is the node corresponding to an edge in the original graph and a vertice node is the node corresponding to a vertice in the original graph. Furhtermore, let $E$ be the set of edges ($i,j$) between the edge nodes $i$ and the vertice nodes $j$ in the NFI graph and $K$ be the clique size that is searched for. Then $\alpha_i = 1$ if node $i$ of the edge nodes is interdicted and $\gamma_j = 1$ if vertice node $j$ is not in the clique. 
\begin{align*}
     \min &  & \sum_{i \in N^E} \alpha_i  \\
     s.t. &  & \alpha_i &\geq \gamma_j & \forall (i,j) \in E \\
          &  & \sum_{j \in N^V} \gamma_j & \leq |N^V| - K \\
          &  & \alpha_i,  && \forall i \in N^E \\ 
          &  & \gamma_j,  & &\forall j \in N^V
\end{align*}
The objective function minimizes the number of not interdicted (in the clique) edge nodes. The first constraint ensures that an all edge nodes $i$ that are connected to an vertice node $j$, which is not in the clique (interdicted) are interdicted as well. The second constraint ensures that the number of vertice nodes that are not in the clique is less than or equal to $|N^V| - K$. 

The idea behind this formulation is that, we do not have to consider all parts of the NFI graph, but only the parts that connect the edge nodes and the vertices nodes. The objective function value of this formulation gives the number of edge nodes that are not interdicted and therefore in the clique, if there are exactly $\binom{K}{2}$ edge nodes that are not interdicted, then we have found a clique of size $K$. This fulfills the two main critera found by Wood to determine a clique of size $K$ in a NFI graph, namely that the number of edge nodes that are interdicted is $|N^E| - \binom{K}{2}$ (all edges not in the clique are interdicted) and the number of verices nodes not interdicted equals $K$, the clique size. 

\section{Results}
We have implemented the above formulation in Python using the Gurobi solver. While we were able to verify the validity of our formulation on small instances, we were not able to solve the larger benchmark instances in a reasonable amount of time. The main reason for this is that the formulation is leveling out the fractional values in the LP-relaxation as much as possible. Therefore, we were not able to find the optimal integer solution. Additionally, we had the problem that only an integer solution with the objective value of $\binom{K}{2}$ would proof the existens of a clique of size $K$, with the objective value of the LP-relaxation not providing any information about the problem. 

\end{document}
