\documentclass{article}
\usepackage{graphicx} % Required for inserting images
\usepackage{amsmath}

\title{Decision Support in Production, Logistics and Supply Chain}
\author{Arezoo Amiri \and Christof Brandstetter \and Tamara Ertl}
\date{\today}

\begin{document}

\maketitle

\section{Introduction}
We only need to add edges with G.add edge(node1, node2, capacity)
It automatically adds nodes

Pyvis does not add nodes automatically

net.toggle physics(True), sometimes makes sense to set to False, It makes the graph draggable

Generate $C_l$ cut where $\gamma = 1$ and generate $C_m$ where $\gamma = u$.
Then calculate the bisection to obtain a new $\hat{\gamma}$. 
Then generate a cut with $\hat{\gamma}$ and do this until we find a cut that has less than $B$ large edges.

% Transform the problem to max clique, 
\section{Notation}
\begin{itemize}
    \item A cut is a partition $V  = S \cup T$ of the nodes of G such that $s \in S$ and $t \in T$
    \item An arc $r \in E$ is in a cut $C = (S,T)$ if $\alpha(r) \in S$ and $\omega(r) \in T$
    \item Arcs having capacity $u$ are called large arcs
    \item Arcs having capacity 1 are called small arcs
    \item $q(C) := \#$(large arcs in $C$)
    \item $p(C) := \#$(small arcs in $C$)
\end{itemize}
\section{What's to do?}
\begin{itemize}
    \item Transform the digraph into a NFI graph
    \begin{itemize}
        \item Create a $s$-node
        \item Create a node for every edge in the digraph
        \item Create edges from $s$ to every \emph{edge-node} $i_x$ with capacity 2m
        \item Create a node for every node in the digraph
        \item Create edges from every \emph{edge-node} to every \emph{node-node} $j_1$ which is connected by the edges with capacity m
        \item Create a node $t$ and connect the \emph{node-nodes} with the $t$ node by edges with capacity m
        \item where $m = |E|$
    \end{itemize}
    \item Implement the bisection algorithm for NFI
    \begin{itemize}
        \item Find minimum cut $C$
        \item Identify the arcs $R \in C$ which are removed from the graph
        \item $C_m$ is optimal if it contains at least $B$ large arcs with $val(C_m) = \text{cap of arcs in cut} - \text{cap of removed arcs}$
        \item Assume: Any minimum cut in $G$ contains at most $B-1$ large arcs
        \item $C_l$ denote a least cut in G, i.e.,  a cut with least possible number of arcs. Then $C_l$ is optimal if it contains at most $B$ large arcs
        \item Assume: Any least cut in $G$ contains at least $B+1$ large arcs
    \end{itemize}
    \item Find in the NFI graph a strategy for NFI with budget $B = |E| - \binom{K}{2} $ that has value $K*m$ to get a clique of size $K$
    \item Transform it back to find the max-clique
    \item Use the bisection algorithm for NFI to find large cliques for the benchmark set
    \begin{itemize}
        \item Suspect, that we have to do this for different K and raise the K's
    \end{itemize}
\end{itemize}
\end{document}
