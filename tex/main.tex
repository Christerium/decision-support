\documentclass{article}
\usepackage{graphicx} % Required for inserting images
\usepackage{amsmath}

\title{Decision Support in Production, Logistics and Supply Chain}
\author{Arezoo Amiri \and Christof Brandstetter \and Tamara Ertl}
\date{\today}

\begin{document}

\maketitle

\section{Introduction}
We only need to add edges with G.add edge(node1, node2, capacity)
It automatically adds nodes

Pyvis does not add nodes automatically

net.toggle physics(True), sometimes makes sense to set to False, It makes the graph draggable

Generate $C_l$ cut where $\gamma = 1$ and generate $C_m$ where $\gamma = u$.
Then calculate the bisection to obtain a new $\hat{\gamma}$. 
Then generate a cut with $\hat{\gamma}$ and do this until we find a cut that has less than $B$ large edges.

% Transform the problem to max clique, 
\section{Notation}
\begin{itemize}
    \item A cut is a partition $V  = S \cup T$ of the nodes of G such that $s \in S$ and $t \in T$
    \item An arc $r \in E$ is in a cut $C = (S,T)$ if $\alpha(r) \in S$ and $\omega(r) \in T$
    \item Arcs having capacity $u$ are called large arcs
    \item Arcs having capacity 1 are called small arcs
    \item $q(C) := \#$(large arcs in $C$)
    \item $p(C) := \#$(small arcs in $C$)
\end{itemize}
\section{What's to do?}
\begin{itemize}
    \item Transform the digraph into a NFI graph
    \begin{itemize}
        \item Create a $s$-node
        \item Create a node for every edge in the digraph
        \item Create edges from $s$ to every \emph{edge-node} $i_x$ with capacity 2m
        \item Create a node for every node in the digraph
        \item Create edges from every \emph{edge-node} to every \emph{node-node} $j_1$ which is connected by the edges with capacity m
        \item Create a node $t$ and connect the \emph{node-nodes} with the $t$ node by edges with capacity m
        \item where $m = |E|$
    \end{itemize}
    \item Implement the bisection algorithm for NFI
    \begin{itemize}
        \item Generate a minimum cut $C_m$ (minimum Cut to the original graph) and get $q(C_m)$ and $p(C_m)$
        \item Generate a least cut $C_l$ (minimum cut of G with the capacity of all arcs set to 1) and get $q(C_l)$ and $p(C_l)$
        \item Calculate $\hat{\gamma}$ and generate the cut with capacity-$\hat{\gamma}$-min-cut $\hat{C}$
        \item If $cap^\gamma(\hat{C}) \leq cap^\gamma(C_l)$ and $q(\hat{C}) \notin \{q(C_l),q(C_m)\}$ then generate two new cuts using $((C_l, \hat{C}), (\hat{C}, C_m))$
        \item otherwise return $\{C_1, C_2\}$
        \item The minimum cut is the smallest set of edges that, when removed, disconnects the graph into two disjoint subgraphs.
        \item Identify the arcs $R \in C$ which are removed from the graph
        \item $C_m$ is optimal if it contains at least $B$ large arcs with $val(C_m) = \text{cap of arcs in cut} - \text{cap of removed arcs}$
        \item Assume: Any minimum cut in $G$ contains at most $B-1$ large arcs
        \item $C_l$ denotes a least cut in G, i.e.,  a cut with least possible number of arcs. Then $C_l$ is optimal if it contains at most $B$ large arcs
        \item Assume: Any least cut in $G$ contains at least $B+1$ large arcs
    \end{itemize}
    \item Find in the NFI graph a strategy for NFI with budget $B = |E| - \binom{K}{2} $ that has value $K*m$ to get a clique of size $K$
    \item Transform it back to find the max-clique
    \item Use the bisection algorithm for NFI to find large cliques for the benchmark set
    \begin{itemize}
        \item Suspect, that we have to do this for different K and raise the K's
    \end{itemize}
\end{itemize}

$R \subseteq E$ is a solution to the u-NFI problem with objective value val(R) which equals the capacity of a minimum s-t-cut in the graph $G_R$.
Minimum s-t-cut of a graph $G_R$ is a minimum cut (cut with least number of arcs, which disjoints s and t) and the capacity of this cut equals the maximum 
flow of the grpah. (Value of a cut is the sum of the capacities of the arcs in the cut) -> cut != removing arcs

So network flow interdiction problem is about finding a subset of arcs $R$ that minimizes the maximum flow from s to t, where all arcs have different capacities 
(do not need to be different but there are several different ones).

The u-NFI has small (1) and large (u) arcs. Here we need to keep in mind, that the val(C) = val($R_C$) = sum of capacity of the $B$ largest arcs


If think the idea of algorithm to find a good solution to the u-NFI is that the we have to compute q-min-cuts 
(cut with smallest capacity in graph G amongst all cuts with exactly q large arcs) for different values of q. So find the minimum cut with minimal capacity
and exactly q large arcs for different values of q and caluclating a q-min-cut is NP-hard.

So we calculate minimum cuts when varying the capacity of the large arcs to obtain cuts with different numbers of q.

If $C^\gamma$ is a capacity-$\gamma$-min-cut for some $\gamma \geq 1$, then it is a q-min-cut for $q = q(C^\gamma)$. This means that if we have a capacity-
$\gamma$-min-cut $C^\gamma$ we have found a q-min-cut with q = q($C^\gamma$) or alternatively, by finding a minimum cut for a certain $\gamma$ we obtain a q-min-cut 
where q equals the number of large arcs in our minimum cut. 

Therefore, we start with $C_l$ and $C_m$ as the q obtained from these are bounds on the optimal q. Then we pick the next $\gamma$ to check for by doing this 
bisection. If this cut has a lower capacity than our lower bound (is below our two lines) we do the bisection again for $C_l$ $\hat{C}$ and $\hat{C}$ $C_m$.
An this we do again and again until $\hat{C}$ has a higher capacity. 

To finish this up, we will not find a single cut, but rather a set of two cuts which will give us a lower and upper bound on the optimal value.


\section{Additional thoughts}
\begin{enumerate}
    \item capacity-$\gamma$-min-cut $(C^\gamma)$ are $q$-min-cut for $q = q(C^\gamma)$
    \item $q(C^{\gamma_1}) \geq q(C^{\gamma_2})$ for $1 \geq \gamma_1 \geq \gamma_2 \geq u$ $\rightarrow \#$ of large arcs is decreasing for increasing $\gamma$
    \item We have bounds on q with $q(C_l)$ and $q(C_m)$
    \item Given two cuts $C_1$ and $C_2$, the next $\hat{\gamma}$ is chosen by the value, when $cap(C_1) = cap(C_2)$ in the graph $G^{\hat{\gamma}}$
    \item If $cap(C^{\hat{\gamma}}) < cap(C_1) = cap(C_2)$ in $G^{\hat{\gamma}}$ bisection is called again for $C_1$ and $\hat{C}$ and $C_2$ and $\hat{C}$
    \item Otherwise the recursion ends and returns $C_1$ and $C_2$
    \item We can use digrpahs for finding the cut as it does not matter for finding the cut
    \item When we have the cut, we try to find the $B$ where $B =  
\end{enumerate}
The target here is to find a capacity-$\gamma$-min-cut with high $\gamma$ (equals low \# of large arcs) wihle reducing a low cut capacity. 

\end{document}
